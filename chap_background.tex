\chapter{Background}
\label{chap:background}

\section{Background}
(I don’t exact know what my thesis will look like, and I don’t have a clear picture about the topic which I am going to do. But, I think Visual Programming Language and Complier are two topics I have to dig into. So, beginning is just a brief background about those two concepts. I do need to figure out what I am going to do in the next two years. And I wish I could enjoy my thesis project.)

\subsection{Visual Programming Language (VPL)}
Visual Programming Language is any programming language that lets users create programs by manipulating program elements graphically rather than by specifying them textually. For different using purposes, there are different softwares or say different languages. For now, we will take a look at two VPLs, LabVIEW and OpenDX.

Laboratory Virtual Instrument Engineering Workbench (LabVIEW) is a programming environment that features a dataflow-based VPL (called G) designed to facilitate development of data acquisition, analysis, display and control applications.\cite{WHITLEY:2001:VPW:2826730.2826856} While among those features, some people did a survey to see how people rate the ability of LabVIEW to transform graphics into computational structures and the usability and accessibility of programming tools in LabVIEW. It turns out that people do love the visual language feature of LabVIEW more that other features.

OpenDX originated as a software product known as IBM Visualization System’s “Visualization Data Explorer”, or Data Explorer, or simply DX. It provides data visualization and analysis. 

\subsection{Compiler}
A compiler is a computer program which could translate source code from a high-level programming language to a lower level language.

\section{Previous Work}

In seminar today, we looked for papers on the ACM Digital Library, about Machine Learning \cite{Tong:2002:SVM:944790.944793}.
