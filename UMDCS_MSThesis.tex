%M.S. Computer Science Thesis
\documentclass{umdthesis}

% This package will nag about old-style LaTeX use.  Feel free to
% uncomment if you care about such issues.  \usepackage[l2tabu,
% orthodox]{nag}

% Note as of 11/18/14
% At the time of this update, the website:
% http://www.olivierverdier.com/posts/2013/07/15/modern-latex/ had
% useful information for Modern LaTeX... some of it is included here.

% Uses new biblatex system rather than traditional bibtex
\usepackage[backref=true, isbn=true, url=true, firstinits=true, maxnames=20, style=numeric, backend=biber,]{biblatex}

% Your references are placed into the .bib file and then specified here:
\addbibresource{UMDCS_MSThesis.bib}

\usepackage{fontspec}
\usepackage{tikz} % Required for flow chart
\usetikzlibrary{positioning,shapes,arrows} % Tikz libraries required for the flow chart in the template
\setmainfont{TimesNewRomanPSMT}
% \setmathfont{CambriaMath}
% \setmainfont{Times}
% \setsansfont{Helvetica}

% This will highlight code examples with nice coloring if you have that in your thesis.
%
% This example uses C++, but there are many nice examples for other languages too.
%
% Nice C++ listings example derived from http://timmurphy.org/2014/01/27/displaying-code-in-latex-documents/
\lstset{
    language=C++, % if you need to change the language...
    frame=tb, % draw a frame at the top and bottom of the code block
    tabsize=4, % tab space width
    showstringspaces=false, % don't mark spaces in strings
    numbers=left, % display line numbers on the left
    commentstyle=\color{green}, % comment color
    keywordstyle=\color{blue}, % keyword color
    stringstyle=\color{red} % string color
}

\RequirePackage{amsmath}
\RequirePackage{amssymb}

% Supplied by Matt Overby - Spring 2014 - This is a nice command that
% lets you mark sections you will come back to later.
\newcommand{\TODO}[1]{\colorbox{yellow}{\textbf{TODO}: #1}}

\title{Your Great Title For Your Thesis}
\author{Penghuan Ni}

\advisor{Professor Advisor Name}

\copyrightyear{2017}

\ackfile{acknowledge}
\dedication{dedication}
\abstractfile{abstract}

\begin{document}

	\frontmatter

        % You could save some space when initially printing (if you
	% really need to) by uncommenting this line to single
	% space
        %
        % \singlespace

        % Ideally the Introduction to your thesis
	\chapter{Introduction}
\label{chap:intro}

\iffalse
It's always good to introduce your (1) problem, (2) why it is interesting, (3) what you did, and (4) roughly, 
how well did it work. You might even have citations in here, as in this paper~\cite{Asawa:2008:TDT}.
\fi

In this paper, we applied the NSGA-II to Urban Design Model to find the Pareto-optimal solutions. Urban design is a complicated problem, it involved many factors, such as sunshine, wind, water area, building height, street width, trees, public space, etc. How do we take into account all those factors becomes a worldwide problem. In the past, people could only make decisions based on their past experience. It turns out that experience could give us some decent results, however, it will never always give us the best result. Human beings are limited and always make mistakes. In this Computer Age, why don’t we seek help from computer. Computer has a lot of advantages compared with human being, it is more reliable, its computation ability is much more than human being. We could simulate this problem with computer and ask computer to solve it for us. 




        % Any background necessary to understand your thesis.  This
        % can also contain the related work too.
	\chapter{Background}
\label{chap:background}

\section{Background}
Based on the previous work of people using NSGA-II on some multi-objective optimization problems\cite{Magnier_2010_Multiobjective}, we believe we can also applied NSGA-II on our Urban Design Model to find the optimal solution. 

\section{Previous Work}

\subsection{Genetic Algorithm(GA)}
After the Genetic Algorithm was introduced in 1975 by John Henry Holland\cite{Holland_1975_Book}, it has been applied in many areas. With mimicking the process of natural selection, GA uses selection, crossover, and mutation to generate solutions and try to find the optimal solution for our problems. 

\subsection{Nondominated Sorting Genetic Algorithm II(NSGA-II)}
NSGA-II\cite{NSGA-II} is a very famous multi-objective optimization algorithm which is wide used  to find the optimal solution nowadays. Compare with the previous NSGA, NSGA-II has improved the computational efficiency, elitism, and sharing parameter. For computational complexity of nondominated sorting, it improve from $O(MN\^{3}) to O(MN^{2}). For WIth the elitism, NSGA-II could not only speed up the GA performance, but also prevent loss of good solutions. Moreover, NSGA II does not need to specify a sharing parameter(\sigma_{share}), which is a requirement for pervious multi-objective evolutionary algorithms.


        % Of course, if you wanted a separate chapter for Related
        % Work, you could make one too.
	% \include{chap_relatedwork}

        % The major content for your thesis!  What did you do and how
        % did you do it!
	\chapter{Implementation}
\label{chap:impl}

\section{Steerable GA Explanation}
Our Steerable Genetic Algorithm take user's feedbacks into account and generate an user preferred solution. Following is the flowchart for our Steerable GA and there are more explanations under the flowchart.

% Define block styles

\tikzstyle{decision} = [diamond, draw, fill=blue!20,
text width=4.5em, text badly centered, node distance=2.5cm, inner sep=0pt]
\tikzstyle{block} = [rectangle, draw, fill=blue!20, node distance=2cm,
text width=6em, text centered, rounded corners, minimum height=1em]
\tikzstyle{line} = [draw, -latex']
\tikzstyle{cloud} = [draw, ellipse, fill=red!20, node distance=3cm, minimum height=2em]

\hspace*{1.5em}\vspace*{1.5em}{
		\begin{tikzpicture}[node distance = 2cm, auto]
		% \hspace*{3em}
		% Place nodes
		\node [block] (init) {start};
		\node [block, text width=10em, below = .8cm of init] (initpop) {generate initial population};
		\node [block, below = .81cm of initpop] (calcfitness) {calculate fitness of each individual };
		%     \node [block, below of=initpop] (evaluate) {evaluate candidate models};
		%     \node [block, left of=evaluate, node distance=3cm] (update) {update model};
		\node [block, below = .8cm of calcfitness] (newpop) {create new population};
		\node [block, right = 1.5cm of newpop] (crossover) {crossover};
		\node [block, above = 1cm of crossover] (selection) {selection};
		\node [block, below = 1cm of crossover] (mutation) {mutation};
		\node [block, below = 1cm of mutation, fill=red!20] (feedback) {feedback};
		\node [decision, below = .8cm of newpop] (decide) {end condition};

		\node [block, below = .8cm of decide, node distance=2cm] (stop) {stop};
		    % Draw edges
	    \path [line] (init) -- (initpop);
	    \path [line] (initpop) -- (calcfitness);
	    \path [line] (calcfitness) -- (newpop);
	    \path [line] ([yshift=0.1cm] newpop.east) -| ([xshift=-.82cm] selection.west) -- (selection.west);
	    \path [line] (selection) -- (crossover);
	    \path [line] (crossover) -- (mutation);
	    \path [line, dashed, draw=red!80] (mutation) -- (feedback);
	    \path [line, dashed, draw=red!80] (feedback) -| ([xshift=-0.8cm] mutation.west);
	    \path [line] (mutation) -| ([xshift=.7cm, yshift=-0.1cm] newpop.east) -- ([yshift=-0.1cm] newpop.east);
	    \path [line] (newpop) -- (decide);
	    \path [line] (decide) -| node [near start] {no} ([xshift=-1.5cm] newpop.west) -- (newpop);
		\path [line] (decide) -- node {yes}(stop);
		\end{tikzpicture}
	}


% NSGA-II is a fast and elitist multiobjective genetic algorithm. The source code we are using is from: \url{http://www.iitk.ac.in/kangal/codes.shtml}. And we are using version 1.1 in our problem. In this section, we will take a first look at how NSGA-II works and get familiar with it.

\subsection{Run and Compile} % Readme, Makefile, global.h, rand.h
% In the code package, there is a Readme file, which tells us some basic information about this code package. The code package is compiled by using Makefile, which has already been provided in the code. This Makefile attempts to compile and link all the existing source files into one single executable. If we want to make some special changes for the code package and do not want to include some of the source code in the future, we should modify the Makefile before we compile the code.

\subsection{Input and Output} % problemdef.c .out file, input folder

\subsection{Operators} %

\section{New Section For Next Important Topic}

\subsection{Algorithm Initialization}
\subsection{Atomic Operations}

You may even need code in your thesis. Here is a way to nicely include code with \LaTeX using the listings package.
\begin{lstlisting}
for (unsigned int idx=0; idx<maxSize; idx++) {
  atomic_add( idx );
}
\end{lstlisting}

\subsection{Programming Style}
\subsubsection{Explaining Fine Detail Here}
\TODO{Make sure to finish this!}

\subsubsection{Last Subsection}


        % It's always good to have a results chapter so you can
        % present how well your ideas and implementations worked
	\chapter{Results}
\label{chap:results}

Your results.  This worked great.  Here's a plot to show how great it worked.  

\begin{figure}[htp] 
\centering
\includegraphics[scale=.4]{images/goodData.png}
\caption{Good data.}
\label{fig:goodData}
\end{figure} 

Figure \ref{fig:goodData}


        % Time to wrap up the thesis with a discussion of your ideas
        % and knowledge that you generated, along with any important
        % insights, or things you learned.  You can include ideas for
        % future extensions and effort here too.
	\include{chap_conclusion}

        % If you need an appendix (or appendices), they can be added here
	\include{appendix}

        % And finally, don't forget the references and bibliography.
        % You can add entries to the file UMDCS_Thesis.bib for your
        % references.  You then need to ``cite'' them in the tex files
        % by using the ~\cite{ReferenceID} tags.
	\newpage
        \printbibliography

\end{document}
