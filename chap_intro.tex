\chapter{Introduction}
\label{chap:intro}

\iffalse
It's always good to introduce your (1) problem, (2) why it is interesting, (3) what you did, and (4) roughly, 
how well did it work. You might even have citations in here, as in this paper~\cite{Asawa:2008:TDT}.
\fi

In this paper, we applied the NSGA-II to Urban Design Model to find the Pareto-optimal solutions. Urban design is a complicated problem, it involved many factors, such as sunshine, wind, water area, building height, street width, trees, public space, etc. How do we take into account all those factors becomes a worldwide problem. In the past, people could only make decisions based on their past experience. It turns out that experience could give us some decent results, however, it will never always give us the best result. Human beings are limited and always make mistakes. In this Computer Age, why don’t we seek help from computer. Computer has a lot of advantages compared with human being, it is more reliable, its computation ability is much more than human being. We could simulate this problem with computer and ask computer to solve it for us. 


